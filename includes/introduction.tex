

 Automatizované sklady predstavujú revolučný krok v logistike a skladovaní. Tieto pokročilé riešenia spájajú moderné technológie, ako sú robotika, umelá inteligencia a automatizácia procesov, s cieľom optimalizovať úlohy spojené so skladovaním a manipuláciou tovaru.
 
V automatizovaných skladoch je centrálnym prvkom robotizované zariadenie, ktoré zabezpečuje manipuláciu s tovarom. Tieto roboty sú schopné efektívne a presne vykonávať rôzne úlohy, vrátane nakladania a vykladania tovaru, presúvania regálov, či dokonca pripravovania objednávok. Vďaka ich schopnosti samostatne sa navigovať a komunikovať s ostatnými zariadeniami v sklade sa celý proces skladovania stáva rýchlejším a efektívnejším.
 

 Slovenská spoločnosť Photoneo/Brightpick ponúka jedinečné riešenia v oblasti automatizácie skladových operácií. Vyvinuli robotické zariadenie Brightpick Autopicker, ktoré kombinuje viacero technologických aplikácií s cieľom poskytnúť efektívne a presné riešenie.
 
 Tento inovatívny systém zahŕňa mobilný podvozok, zdvíhaciu plošinu na manipuláciu s krabicami, 3D kameru a robotické rameno. Zdvíhacia plošina je vybavená mechanizmom na manipuláciu s krabicami, umožňujúcim ich presun a usporiadanie na regáloch. Po vybratí krabice s daným tovarom 3D kamera skenuje obsah a získava dôležité informácie o veľkosti a pozíciách jednotlivých položiek.
 
 Samotné robotické rameno potom prevádza presuny tovaru z krabice s tovarom do krabice s objednávkou. Všetky tieto operácie sú realizované s vysokou presnosťou a efektivitou vďaka spojeniu moderných technológií a inteligentného riadenia.
 Tento komplexný prístup k automatizácii skladových operácií umožňuje firmám zvýšiť produktivitu, znížiť chybovosť a lepšie reagovať na dynamické potreby trhu.
 
Táto práca sa zameriava na robotické rameno so špecifickou kinematickou štruktúrou SCARA. Tieto typy robotov sú obzvlášť známe svojou schopnosťou vykonávať opakujúce sa úlohy s vysokou presnosťou a efektívnosťou, čo ich činí neodmysliteľnou súčasťou moderných výrobných zariadení.

V súčasnom riešení je pracovný priestor navrhnutý tak, aby nedochádzalo k žiadnym kolíziám medzi prenášaným tovarom a okolitým prostredím alebo samotným robotom. Avšak, môžu nastať situácie, keď sú rozmery prenášaného tovaru väčšie, a preto nie je možné dosiahnuť bezkolízný pracovný priestor. Takéto prípady vyžadujú ďalšie opatrenia a riešenia, aby sa minimalizovali riziká kolízií a zabezpečila bezpečná manipulácia s tovarom.

 Cieľom práce je navrhnúť taký algoritmus plánovania trajektórie, ktorý bude zohľadňovať prostredie aj reálny tvar a rozmery   prenášaného tovaru. Taktiež sa zameriame na samotnú kinematickú štruktúru robota, ktorá priamo ovplyvňuje možnosti algoritmu.
 Našim cieľom je identifikovať najefektívnejšiu kinematickú štruktúru pre danú aplikáciu a zvážiť výhody a nevýhody rôznych prístupov.  
 

 
 
 
 