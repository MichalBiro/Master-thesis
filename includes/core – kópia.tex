
Chcel by som sa poďakovať môjmu vedúcemu práce ...
\section{Ukážka glossaries}

\noindent Verzia FEIstyle 1.5 používa glossary\footnote{\url{https://www.ctan.org/pkg/glossaries?lang=en}} balík.
\acrfull{cdma} je dlhá skratka naopak \acrshort{gsm} je skratka v krátkej forme.

\subsection{Lorem Ipsum}
Tento text bol prevzaty zo stranky \url{www.lipsum.com} \cite{lipsum}, preto ho treba riadne odcitovat.

Lorem Ipsum is simply dummy text of the printing and typesetting industry. Lorem Ipsum has been the industry's standard dummy text ever since the 1500s, when an unknown printer took a galley of type and scrambled it to make a type specimen book. It has survived not only five centuries, but also the leap into electronic typesetting, remaining essentially unchanged. It was popularised in the 1960s with the release of Letraset sheets containing Lorem Ipsum passages, and more recently with desktop publishing software like Aldus PageMaker including versions of Lorem Ipsum \cite{lipsum}.

Why do we use it? It is a long established fact that a reader will be distracted by the readable content of a page when looking at its layout. The point of using Lorem Ipsum is that it has a more-or-less normal distribution of letters, as opposed to using 'Content here, content here', making it look like readable English \cite{lipsum}. Many desktop publishing packages and web page editors now use Lorem Ipsum as their default model text, and a search for 'lorem ipsum' will uncover many web sites still in their infancy. Various versions have evolved over the years, sometimes by accident, sometimes on purpose (injected humour and the like).

Where does it come from? Contrary to popular belief, Lorem Ipsum is not simply random text. It has roots in a piece of classical Latin literature from 45 BC, making it over 2000 years old. Richard McClintock, a Latin professor at Hampden-Sydney College in Virginia, looked up one of the more obscure Latin words, consectetur, from a Lorem Ipsum passage, and going through the cites of the word in classical literature, discovered the undoubtable source \cite{lipsum}. Lorem Ipsum comes from sections 1.10.32 and 1.10.33 of "de Finibus Bonorum et Malorum" (The Extremes of Good and Evil) by Cicero, written in 45 BC. This book is a treatise on the theory of ethics, very popular during the Renaissance. The first line of Lorem Ipsum, "Lorem ipsum dolor sit amet..", comes from a line in section 1.10.32 \cite{lipsum}.

\section{Recitácia}
Citujem všetky zdroje v \textbf{bibliography.bib}, \cite{t00, t01, t02, t03, kniha, kniha2, kniha3, small, big, cs, koll, kap, tug, knuth, zbornik, prispevok}. \newline Good luck.

\section{Rovnice}
Pytagorova veta je definovana vztahom:
\begin{equation}
    a^2 + b^2 = c^2
    \label{eq:pyth}
\end{equation}
Dosadenim \eqref{eq:pyth} do linearneho modelu
\begin{align}
    \dot{\mathbf{x}} &= \mathbf{A} \mathbf{x} + \mathbf{B} \mathbf{u} \\
    \mathbf{y} &= \mathbf{C} \mathbf{x} + \mathbf{D} \mathbf{u}
    \label{eq:model}
\end{align}
zistime, ze tento \LaTeX template funguje, dokonca mozeme vkladat aj inline rovnice $D = b^2 - 4ac$.

% \subsection{Podsekcia}
% \begin{equation}
%     E = m c^2
%     \label{eq:emc2}
% \end{equation}

\section{Možnosti anonymizácie}
\noindent Anonymizácia znamená zmena alebo úprava údajov tak, aby sa podľa nich nedala jednoznačne určiť osoba, ktorej tieto údaje patria \cite{t01}. Existuje niekoľko spôsobov, ktorými môžeme dosiahnuť rôznu úroveň anonymizácie na internete: od mazania cookies súborov po ukončení prehliadania webových stránok až po používanie operačných systémov, ktoré sú na anonymite založené; od bezplatných možností až po komerčné verzie.  
\newline Nasleduje priblíženie niektorých možnosti anonymizácie.

\subsection{Súkromné prehliadanie}
\noindent Najpoužívanejšie internetové prehliadače súčasnosti majú v sebe zabudovanú funkcionalitu, ktorá dokáže čiastočne anonymizovať prístup na internet. Táto funkcionalita blokuje ukladanie navštívených stránok do histórie a nezaznamenáva súbory, ktoré sa stiahnu z~internetu. \acrshort{sw} a \acrlong{hw} sú skratky.

\begin{table}[!htbp]
\caption{Moduly a ich funkcie pri anonymizácii}
\label{modulyVlastnosti}
\begin{center}
\begin{tabular}{p{4cm}|c|c|c|c|c|c|c|c|c|c|c|c|c|c|c}
& \multicolumn{14}{c}%
	 {\textbf{Funkcia}}\\ \hline
&&&& & &\multicolumn{8}{c}%
	 {Modifikácia}\\ 
\textbf{Modul} &\begin{sideways} zobrazenie hlavičky \end{sideways} &\begin{sideways} blokovanie skriptov \end{sideways} &\begin{sideways} zmena IP \end{sideways} & \begin{sideways} zmena lokalizácie \end{sideways} & \begin{sideways} zmazanie/blokovanie cookies \end{sideways} & \begin{sideways} blokovanie trackerov \end{sideways}  & \begin{sideways} popis \end{sideways} & \begin{sideways}používateľský agent\end{sideways} & \begin{sideways} kódové označenie prehliadača \end{sideways} & \begin{sideways} názov prehliadača \end{sideways} & \begin{sideways} verzia prehliadača \end{sideways} & \begin{sideways} platforma \end{sideways} & \begin{sideways} výrobca prehliadača \end{sideways} & \begin{sideways} označenie výrobcu prehliadača \end{sideways} \\ \hline
User agent switcher & & & & & &  & X & X & X & X & X & X & X & X  \\ \hline
Ghostery &  && & & X & X &  &  & & & & & & \\  \hline
Better privacy && &  & & X &  &  &  & & & & & & \\  \hline
Anonymox &  && X & X & X &  & X & X & & & & & & \\  \hline
Modify headers & & &  &  & X &  &  & X &  &  &  & & &  \\  \hline
Request policy & & &  &  & & X  &  &  &  &  &  & & &   \\  \hline
Live HTTP headers & X& &  &  & &  &  &  &  &  &  & & &   \\  \hline
User agent awitcher for chrome & & &  &  & &  & X & X &  &  &  & & &   \\  \hline
Header hacker & & &  &  & &  & X & X & X & X & X & X & X & X    \\  \hline
Mod header & & &  &  & &  & X & X & X & X & X & X & X & X    \\  \hline
Script no & &X &  &  & &  &  &  &  &  &  &  &  &     \\  \hline
No script & &X &  &  & &  &  &  &  &  &  &  &  &     \\  \hline
Proxify it & & &X  & X & &  &  &  &  &  &  &  &  &     \\  \hline
I'm not here & & &  & X & &  &  &  &  &  &  &  &  &     \\  \hline
Get anonymous personal edition & &X &X &X &X&X &  &  &  &  &  &  &  &     \\  \hline
Anonymous browsing toolbar & & & X & X & &  &  &  &  &  &  &  &  &     \\  \hline
Easy hide your IP and surf anonymously & & & X & X& &  &  & X & X & X & X &  &  &     \\  \hline
\end{tabular}
\end{center}
\end{table}

\subsection{Anonymná sieť}
\noindent Anonymná sieť je sieť serverov, medzi ktorými dáta prechádzajú šifrované. V anonymných sieťach dáta prechádzajú z počítača používateľa, odkiaľ bola požiadavka poslaná, cez viaceré proxy smerovače, z ktorých každý správu doplní o smerovanie a zašifruje vlastným kľúčom. Cesta od ...


\subsection{Funkcionalita}
\noindent  Rozšírenie tiež okrem splnenia špecifikácie malo pre prehľadnosť a overenie funkčnosti zobrazovať údaje, ktoré boli na server odoslané. Zoznam údajov odoslaných na server, sa mal ukladať do krátkodobej histórie, aby nemal používateľ k dispozícií len najnovšie údaje, ale aj údaje odoslané v nejakom časovom období. Nejaky listing z priloh \ref{lst:sublime}.

\subsubsection{Funkcionalita2}
\noindent Samozrejmosťou bolo nastavenie zapnutia rozšírenia pri štarte, prípadne interval zmeny odosielaných údajov.

\subsection{Vzhľad}
\noindent Dôležitou požiadavkou kladenou na rozšírenie bolo príjemné používateľské rozhranie. Z~tohto dôvodu malo rozšírenie obsahovať zoznam modifikovaných vlastností a tlačidlo pre prístup k nastaveniam rozšírenia v jednoduchej a praktickej forme. Predpokladaný vzhľad je zobrazený na obrázku č. \ref{vzhladobr}.
\begin{figure}[!htbp]
  \centering
  \includegraphics[width=8cm]{img/vzhlad.png}
  \caption{Predpokladaný vzhľad rozšírenia}
  \label{vzhladobr}
\end{figure}	 
\noindent Dôležitou požiadavkou kladenou na rozšírenie bolo príjemné používateľské rozhranie.\cite{t00} Z~tohto dôvodu malo rozšírenie obsahovať zoznam modifikovaných vlastností a tlačidlo pre prístup k nastaveniam rozšírenia v jednoduchej a praktickej forme. Predpokladaný vzhľad je zobrazený na obrázku č. \ref{vzhladobr}.

\begin{algorithm}
\scriptsize
\begin{algorithmic}
 \STATE <text>
 \IF{<condition>} \STATE {<text>} \ELSE \STATE{<text>} \ENDIF
 \IF{<condition>} \STATE {<text>} \ELSIF{<condition>} \STATE{<text>} \ENDIF
 \FOR{<condition>} \STATE {<text>} \ENDFOR
 \FOR{<condition> \TO <condition> } \STATE {<text>} \ENDFOR
 \FORALL{<condition>} \STATE{<text>} \ENDFOR
 \WHILE{<condition>} \STATE{<text>} \ENDWHILE
 \REPEAT \STATE{<text>} \UNTIL{<condition>}
 \LOOP \STATE{<text>} \ENDLOOP
 \REQUIRE <text>
 \ENSURE <text>
 \RETURN <text>
 \PRINT <text>
 \COMMENT{<text>}
 \AND, \OR, \XOR, \NOT, \TO, \TRUE, \FALSE
\end{algorithmic}
\caption{Ukážka príkazov pre algorithmic}  
\label{alg:preview}  
\end{algorithm}

\begin{lstlisting}[
  caption={Ukážka algoritmu},
  label={lst:main-c},
  language=c
]
/* Hello World program */

#include<stdio.h>

struct cpu_info {
    long unsigned utime, ntime, stime, itime;
    long unsigned iowtime, irqtime, sirqtime;
};

main()
{
    printf("Hello World");
}
\end{lstlisting}